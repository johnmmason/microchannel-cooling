\documentclass{article}

\usepackage{graphicx}
\usepackage{geometry}
\usepackage{amsmath}

\geometry{letterpaper, margin=1in}
\parindent 0pt

\title{The naive method for evaluating a microchannel cooling solution}
\author{The University of Texas at Austin, Senior Design Spring 2023, Group 3}
\date{March 26, 2023}

\begin{document}

\maketitle

The Naive Method function allows the user to calculate the heat flux $q$, pressure loss $dP$, and outlet temperature $T_{out}$ for a simple microchannel cooler which consists of straight rectangular channels.

\section{Input}

The Naive Method function takes the following input

\begin{enumerate}

\item Channel Length, $L$ [$m$]
\item Channel Width, $W$ [$m$]
\item Channel Depth, $D$ [$m$]
\item Fluid Density, $\rho$ [$kg/m^3$]
\item Fluid Viscosity, $\mu$ [$Pa*s$]
\item Fluid Specific Heat, $c_p$ [$J/(kg*K)$]
\item Fluid Thermal Conductivity, $k$ [$W/(m*K)$]
\item Fluid Inlet Temperature, $T_{in}$ [$K$]
\item Wall Temperature, $T_w$ [$K$]
\item Flow Rate, $Q$ [$uL/min$]

\end{enumerate}

\section{Describing the Flow}

The cross-sectional area [$m^2$]

\begin{equation}
	A = WD
\end{equation}

The wetted perimeter [$m$]

\begin{equation}
	P = 2(W+D)
\end{equation}

The hydraulic diameter [$m$]

\begin{equation}
	D_h = 4 \frac{A}{P}
\end{equation}

The fluid velocity [$m/s$]

\begin{equation}
	v = \frac{Q}{A} * 1.67 \times 10^{-11}
\end{equation}

where the constant $1.67 \times 10^{-11}$ ensures dimensional consistency.

\subsection{Dimensionless Flow Parameters}

The Reynolds number for the flow [$ul$]

\begin{equation}
	Re = \frac{ \rho v D_h }{ \mu }
\end{equation}

where $\rho$ is the density of the fluid, $v$ is the speed of the flow, $D_h$ is the hydraulic diameter, and $\mu$ is the viscosity of the fluid. \\

The Prandtl number for the flow [$ul$]

\begin{equation}
	Pr = \frac{c_p\mu}{k}
\end{equation}

where $c_p$ is the specific heat of the fluid, $\mu$ is the viscosity of the fluid, and $k$ is the thermal conductivity of the fluid. \\

The Nusselt number for the flow [$ul$] is calculated using the Reynolds number and the Prandtl number. This is an empirical equation, for fully developed laminar flow in a circular pipe.

\begin{equation}
	Nu = 0.023 * Re^{4/5} * Pr^{1/3}
\end{equation}

\subsection{Heat Transfer}

The heat transfer coefficient, $h$ [$W/(m^2*K)$]

\begin{equation}
	h = \frac{Nu*k}{D_h}
\end{equation}

where $Nu$ is the Nusselt number, $k$ is the fluid thermal conductivity, and $D_h$ is the hydraulic diameter. \\

The heat flux, $q$ [$W/m^2$]

\begin{equation}
	q = h * (T_w - T_{in})
\end{equation}

where $h$ is the heat transfer coefficient, $T_w$ is the wall temperature, and $T_in$ is the inlet temperature.

\subsection{Pressure Loss}

The pressure loss, $dP$ [$Pa$]

\begin{equation}
	dP = \frac{ f L \rho v^2 }{2D}
\end{equation}

where $f$, the Fanning friction factor, is $(64/Re)$, $L$ is the channel length, $\rho$ is the fluid density, $v$ is the fluid velocity, and $D$ is the channel depth.

\subsection{Outlet Temperature}

The outlet temperature, $T_{out}$ [$K$]

\begin{equation}
	T_{out} = T_{in} + \frac{q}{Q \rho c_p * 1.67 \times 10_{-4}}
\end{equation}

where $T_{in}$ is the inlet temperature, $q$ is the heat flux, $Q$ is the flow rate, $\rho$ is the fluid density, and $c_p$ is the specific heat of the fluid.  The constant $1.67 \times 10_{-4}$ ensures dimensional consistency.

\section{References}
Primary reference: Ch.3 (Single-Phase Liquid Flow In Minichannels and Microchannels) of Kandilikar, S. Heat Transfer and fluid flow in minichannels and microchannels.

\end{document}