\documentclass{article}

\usepackage{graphicx}
\usepackage{geometry}
\usepackage{amsmath}

\geometry{letterpaper, margin=1in}
\parindent 0pt

\title{The modified naive method for evaluating a microchannel cooling solution}
\author{The University of Texas at Austin, Senior Design Spring 2023, Group 3}
\date{April 10, 2023}

\begin{document}

\maketitle

The Naive Method function allows the user to calculate the heat flux $q$, pressure loss $dP$, and outlet temperature $T_{out}$ for a simple microchannel cooler which consists of straight rectangular channels.

\section{Input}

The Naive Method function takes the following input

\begin{enumerate}

\item Channel Length, $L$ [$m$]
\item Channel Width, $W$ [$m$]
\item Channel Height, $H$ [$m$]
\item Fluid Density, $\rho$ [$kg/m^3$]
\item Fluid Viscosity, $\mu$ [$Pa*s$]
\item Fluid Specific Heat, $c_p$ [$J/(kg*K)$]
\item Fluid Thermal Conductivity, $k$ [$W/(m*K)$]
\item Fluid Inlet Temperature, $T_{in}$ [$K$]
\item Wall Temperature, $T_w$ [$K$]
\item Flow Rate, $Q$ [$m^3/s$]

\end{enumerate}

\section{Describing the Flow}

The cross-sectional area [$m^2$]

\begin{equation}
	A = WH
\end{equation}

The wetted perimeter [$m$]

\begin{equation}
	P = 2(W+D)
\end{equation}

The hydraulic diameter [$m$]

\begin{equation}
	D_h = 4 \frac{A}{P}
\end{equation}

The fluid velocity [$m/s$]

\begin{equation}
	v = Q/A
\end{equation}

\subsection{Dimensionless Flow Parameters}

The Reynolds number for the flow [$ul$]

\begin{equation}
	Re = \frac{ \rho v D_h }{ \mu }
\end{equation}

where $\rho$ is the density of the fluid, $v$ is the speed of the flow, $D_h$ is the hydraulic diameter, and $\mu$ is the viscosity of the fluid. \\

The Prandtl number for the flow [$ul$]

\begin{equation}
	Pr = \frac{c_p\mu}{k}
\end{equation}

where $c_p$ is the specific heat of the fluid, $\mu$ is the viscosity of the fluid, and $k$ is the thermal conductivity of the fluid. \\

The Nusselt number for the flow [$ul$] is calculated using the Reynolds number and the Prandtl number. This is an empirical relation known as the Dittus and Boelter equation, for fully developed flow in a circular pipe.

\begin{equation}
	Nu = 0.023 * Re^{0.8} * Pr^{0.4}
\end{equation}

\subsection{Heat Transfer}

The heat transfer coefficient, $h$ [$W/(m^2*K)$]

\begin{equation}
	h = \frac{Nu*k}{D_h}
\end{equation}

where $Nu$ is the Nusselt number, $k$ is the fluid thermal conductivity, and $D_h$ is the hydraulic diameter. \\

For each $\delta L$ along the length $L$ of the channel,

\begin{equation}
	T_{out} = \frac{ \delta E }{ \rho Q c_p } + T_{in}
\end{equation}

where $\delta E$ is

\begin{equation}
	\delta E = 4 h W \delta L (T_w - T_{in})
\end{equation}

Finally, the heat flux $q$ [$w/m^2$] can be calculated as 

\begin{equation}
	q = \frac{1}{4WL} \sum \delta E 
\end{equation}

\subsection{Pressure Loss}

The pressure loss, $dP$ [$Pa$]

\begin{equation}
	dP = \frac{ f L \rho v^2 }{2D}
\end{equation}

where $f$, the Fanning friction factor, is $(64/Re)$, $L$ is the channel length, $\rho$ is the fluid density, $v$ is the fluid velocity, and $D$ is the channel depth.

\end{document}